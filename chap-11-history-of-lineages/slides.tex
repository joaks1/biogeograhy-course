\documentclass[table]{beamer}
\usepackage{beamerthemesplit}
\usetheme{boxes}
\usecolortheme{seahorse}
% \useinnertheme{myboxes}
% \usepackage{amsmath}
% \usepackage[fleqn]{amsmath}
\usepackage{ifthen}
\usepackage{xspace}
\usepackage{multirow}
\usepackage{booktabs}
\usepackage{xcolor}
\usepackage[style=nature]{biblatex}
\newrobustcmd*{\footlessfullcite}{\AtNextCite{\renewbibmacro{title}{}\renewbibmacro{in:}{}}\footfullcite}
\AtEveryBibitem{\clearfield{month}}
\AtEveryCitekey{\clearfield{month}}

\usepackage{hyperref}
\hypersetup{pdfborder={0 0 0}, colorlinks=true, urlcolor=blue, linkcolor=blue, citecolor=blue}

% \usepackage[format=plain, labelsep=period, justification=raggedright, singlelinecheck=true, skip=2pt, font=sf]{caption}
\usepackage{caption}
\captionsetup[figure]{font=scriptsize, labelformat=empty, textformat=simple, justification=centering, skip=2pt}

% Make all footnotes smaller
\renewcommand{\footnotesize}{\scriptsize}

\definecolor{myGray}{gray}{0.9}
\colorlet{rowred}{red!30!white}

\setbeamertemplate{blocks}[rounded][shadow=true]

\setbeamercolor{defaultcolor}{bg=structure!30!normal text.bg,fg=black}
\setbeamercolor{block body}{bg=structure!30!normal text.bg,fg=black}
\setbeamercolor{block title}{bg=structure!50!normal text.bg,fg=black}

\newenvironment<>{varblock}[2][\textwidth]{%
  \setlength{\textwidth}{#1}
  \begin{actionenv}#3%
    \def\insertblocktitle{#2}%
    \par%
    \usebeamertemplate{block begin}}
  {\par%
    \usebeamertemplate{block end}%
  \end{actionenv}}

\newenvironment{displaybox}[1][\textwidth]
{
    \centerline\bgroup\hfill
    \begin{beamerboxesrounded}[lower=defaultcolor,shadow=true,width=#1]{}
}
{
    \end{beamerboxesrounded}\hfill\egroup
}

\newenvironment{onlinebox}[1][4cm]
{
    \newbox\mybox
    \newdimen\myboxht
    \setbox\mybox\hbox\bgroup%
        \begin{beamerboxesrounded}[lower=defaultcolor,shadow=true,width=#1]{}
    \centering
}
{
    \end{beamerboxesrounded}\egroup
    \myboxht\ht\mybox
    \raisebox{-0.25\myboxht}{\usebox\mybox}\hspace{2pt}
}

\newenvironment{mydescription}{
    \begin{description}
        \setlength{\leftskip}{-1.5cm}}
    {\end{description}}

\newenvironment{myitemize}{
    \begin{itemize}
        \setlength{\leftskip}{-.3cm}}
    {\end{itemize}}

% define formatting for footer
\newcommand{\myfootline}{%
    {\it
    \insertshorttitle
    \hspace*{\fill} 
    \insertshortauthor, \insertshortinstitute
    % \ifx\insertsubtitle\@empty\else, \insertshortsubtitle\fi
    \hspace*{\fill}
    \insertframenumber/\inserttotalframenumber}}

% set up footer
\setbeamertemplate{footline}{%
    \usebeamerfont{structure}
    \begin{beamercolorbox}[wd=\paperwidth,ht=2.25ex,dp=1ex]{frametitle}%
        \Tiny\hspace*{4mm}\myfootline\hspace{4mm}
    \end{beamercolorbox}}

% remove navigation bar
\beamertemplatenavigationsymbolsempty


% \newcommand{\change}[1]{{\color{blue} #1}\xspace}
\newcommand{\change}[1]{{\color{black} #1}\xspace}

\makeatletter
\newcommand*{\rom}[1]{\expandafter\@slowromancap\romannumeral #1@}
\makeatother

\newcommand{\myHangIndent}{\hangindent=5mm}

\newcommand{\citationNeeded}{\textcolor{magenta}{\textbf{[CITATION NEEDED!]}}\xspace}
\newcommand{\tableNeeded}{\textcolor{magenta}{\textbf{[TABLE NEEDED!]}}\xspace}
\newcommand{\figureNeeded}{\textcolor{magenta}{\textbf{[FIGURE NEEDED!]}}\xspace}
\newcommand{\highLight}[1]{\textcolor{magenta}{\MakeUppercase{#1}}}

\newcommand{\editorialNote}[1]{\textcolor{red}{[\textit{#1}]}}
\newcommand{\ignore}[1]{}
\newcommand{\addTail}[1]{\textit{#1}.---}
\newcommand{\super}[1]{\ensuremath{^{\textrm{#1}}}}
\newcommand{\sub}[1]{\ensuremath{_{\textrm{#1}}}}
\newcommand{\dC}{\ensuremath{^\circ{\textrm{C}}}}
\newcommand{\tb}{\hspace{2em}}

\providecommand{\e}[1]{\ensuremath{\times 10^{#1}}}

\newcommand{\mthnote}[2]{{\color{red} #2}\xspace}
\newcommand{\cwlnote}[2]{{\color{orange} #2}\xspace}

\newcommand{\ifTwoArgs}[3]{\ifthenelse{\equal{#1}{}\or\equal{#2}{}}{}{#3}\xspace}
\newcommand{\ifArg}[2]{\ifthenelse{\equal{#1}{}}{}{#2}\xspace}

\newcommand{\divTime}[1]{\ensuremath{\tau_{#1}}\xspace}
\newcommand{\divTimeVector}{\ensuremath{\boldsymbol{\divTime{}}}\xspace}
\newcommand{\divTimeIndex}[1]{\ensuremath{t_{#1}}\xspace}
\newcommand{\divTimeIndexVector}{\ensuremath{\mathbf{\divTimeIndex{}}}\xspace}
\newcommand{\divTimeMap}[1]{\ensuremath{T_{#1}}\xspace}
\newcommand{\divTimeMapVector}{\ensuremath{\mathbf{\divTimeMap{}}}\xspace}
\newcommand{\divTimeScaled}[2]{\ensuremath{\mathcal{T}_{#1\protect\ifTwoArgs{#1}{#2}{,}#2}}\xspace}
\newcommand{\divTimeScaledVector}{\ensuremath{\mathbf{\divTimeScaled{}{}}}\xspace}
\newcommand{\divTimeMean}{\ensuremath{\bar{\divTimeMap{}}}\xspace}
\newcommand{\divTimeVar}{\ensuremath{s^{2}_{\divTimeMap{}}}\xspace}
\newcommand{\divTimeDispersion}{\ensuremath{D_{\divTimeMap{}}}\xspace}
\newcommand{\divTimeNum}{\ensuremath{\lvert \divTimeVector \rvert}\xspace}
\newcommand{\demographicParams}[1]{\ensuremath{\Theta_{#1}}\xspace}
\newcommand{\demographicParamVector}{\ensuremath{\mathbf{\demographicParams{}}}\xspace}
\newcommand{\popSampleSize}[2]{\ensuremath{n_{#1\protect\ifTwoArgs{#1}{#2}{,}#2}}}
\newcommand{\gammaShape}[1]{\ensuremath{a_{#1}}\xspace}
\newcommand{\gammaScale}[1]{\ensuremath{b_{#1}}\xspace}
\newcommand{\betaA}[1]{\ensuremath{a_{#1}}\xspace}
\newcommand{\betaB}[1]{\ensuremath{b_{#1}}\xspace}
\newcommand{\integerPartitionSet}[1]{\ensuremath{a({#1})}\xspace}
\newcommand{\integerPartitionNum}[1]{\ensuremath{\lvert \integerPartitionSet{#1} \rvert}\xspace}
\newcommand{\concentrationParam}{\ensuremath{\chi}\xspace}
\newcommand{\stirlingFirst}[2]{\ensuremath{c(#1, #2)}\xspace}
\newcommand{\descendantThetaMean}[1]{\ensuremath{\bar{\theta}_{D\protect\ifArg{#1}{,}#1}}\xspace}
\newcommand{\numPriorSamples}{\ensuremath{\mathbf{n}}\xspace}
\newcommand{\paramSampleVector}[1]{\ensuremath{\Lambda_{#1}}\xspace}
\newcommand{\paramSampleMatrix}{\ensuremath{\boldsymbol{\paramSampleVector{}}}\xspace}
\newcommand{\ordered}{\ensuremath{\circ}\xspace}
\newcommand{\modelDPP}{\ensuremath{M_{DPP}}\xspace}
\newcommand{\modelDPPOrdered}{\ensuremath{M^{\ordered}_{DPP}}\xspace}
\newcommand{\modelUniform}{\ensuremath{M_{Uniform}}\xspace}
\newcommand{\modelUshaped}{\ensuremath{M_{Ushaped}}\xspace}
\newcommand{\modelOld}{\ensuremath{M_{msBayes}}\xspace}
\newcommand{\priorDPP}[1]{\ensuremath{DP(\concentrationParam #1)}\xspace}
\newcommand{\priorUniform}{\ensuremath{DU\{\integerPartitionSet{\npairs{}}\}}\xspace}
\newcommand{\priorOld}{\ensuremath{DU\{1, \ldots, \npairs{}\}}\xspace}
\newcommand{\powerSeriesOld}{\ensuremath{\mathcal{M}_{msBayes}}\xspace}
\newcommand{\powerSeriesUniform}{\ensuremath{\mathcal{M}_{Uniform}}\xspace}
\newcommand{\powerSeriesExp}{\ensuremath{\mathcal{M}_{Exp}}\xspace}
\newcommand{\empModelOld}{\ensuremath{\mathbf{M}_{msBayes}}\xspace}
\newcommand{\empModelUniform}{\ensuremath{\mathbf{M}_{Uniform}}\xspace}
\newcommand{\empModelDPP}{\ensuremath{\mathbf{M}_{DPP}}\xspace}
\newcommand{\empModelDPPInform}{\ensuremath{\mathbf{M}^{inform}_{DPP}}\xspace}
\newcommand{\empModelDPPSimple}{\ensuremath{\mathbf{M}^{simple}_{DPP}}\xspace}
\newcommand{\npModelDPP}{\ensuremath{\mathbb{M}_{DPP}}\xspace}
\newcommand{\npModelDPPOrdered}{\ensuremath{\mathbb{M}^{\ordered}_{DPP}}\xspace}


\bibliography{../bib/references}

\title[History of Lineages]{History of Lineages}
\subtitle{Chapter 11}

\author[J.\ Oaks]{
    Jamie Oaks\inst{1}
}
\institute[University of Washington]{
    \inst{1}%
        Kincaid Hall 524 \hspace{1em} \href{mailto:joaks1@gmail.com}{\texttt{joaks1@gmail.com}}
}

% \date{\today}
\date{April 11, 2014}

\begin{document}

% \maketitle
\begin{frame}
    \begin{columns}[c]
        \column{.5\textwidth}
            \maketitle
        \column{.5\textwidth}
            \begin{figure}
                \begin{center}
                \includegraphics[width=\textwidth]{../images/darwin-tol-copyright-boris-kulikov-2007.jpg}
                \caption{\tiny \copyright~2007 Boris Kulikov}
                \end{center}
            \end{figure}
    \end{columns}
\end{frame}

\begin{frame}
    \frametitle{Acknowledgements}
\end{frame}

\begin{frame}
    \frametitle{Bergmann's Rule}
    \begin{columns}[c]
        \column{.5\textwidth}
            \begin{onlyenv}<1->
            \begin{center}
            \begin{tikzpicture}
            [scale=0.35,auto=left,every node/.style={circle}]%,fill=blue!20}]
              \draw [<->,thick] (0,10) -- (0,0) -- (14,0);
              \node [label=below: {\sffamily Latitude}](xl) at (7.5,0.5) {};
              \node [label=left: {\sffamily Mass}](yl) at (0,5) {};
              \node [inode, label={[label distance=-5pt]0:\small\sffamily\bfseries A}](a) at (1,1) {};
              \node [inode, label={[label distance=-5pt]0:\small\sffamily\bfseries B}](d) at (2.5,0.8) {};
              \node [inode, label={[label distance=-5pt]0:\small\sffamily\bfseries C}](c) at (2,2.5) {};
              \node [inode, color=blue, label={[label distance=-5pt]0:\small\sffamily\bfseries D}](e) at (7,7.5) {};
              \node [inode, color=blue, label={[label distance=-5pt]0:\small\sffamily\bfseries E}](f) at (8.1,9.1) {};
              \node [inode, color=blue, label={[label distance=-5pt]0:\small\sffamily\bfseries F}](b) at (10,10) {};
            
            \end{tikzpicture}
            \end{center}
            \end{onlyenv}
        \column{.5\textwidth}
            \begin{onlyenv}<2>
            \begin{center}
            \begin{tikzpicture}
            [scale=0.35,auto=left,every node/.style={circle}]%,fill=blue!20}]
              \node [inode, color=black, label={[label distance=-5pt]90:\small\sffamily\bfseries A}](a) at (1, 7) {};
              \node [inode, color=black, label={[label distance=-5pt]90:\small\sffamily\bfseries B}](b) at (3, 7) {};
              \node [inode, color=black, label={[label distance=-5pt]90:\small\sffamily\bfseries C}](c) at (5, 7) {};
              \node [inode, color=blue, label={[label distance=-5pt]90:\small\sffamily\bfseries D}](d) at (7, 7) {};
              \node [inode, color=blue, label={[label distance=-5pt]90:\small\sffamily\bfseries E}](e) at (9, 7) {};
              \node [inode, color=blue, label={[label distance=-5pt]90:\small\sffamily\bfseries F}](f) at (11, 7) {};

              \draw [very thick] (a) -- (3,5) -- (4,6) -- (c);
              \draw [very thick] (4,6) -- (3.5, 6.5);
              \draw [very thick] (3.5, 6.5) -- (b);
              \draw [very thick, draw=blue] (d) -- (9,5) -- (10,6) -- (f);
              \draw [very thick, draw=blue] (10,6) -- (9.5, 6.5);
              \draw [very thick, draw=blue] (9.5, 6.5) -- (e);
              \draw [very thick] (3,5) -- (6,-4) -- (7.5, 0.5);
              \draw [very thick, draw=blue] (7.5, 0.5) -- (9, 5);
              \draw [very thick] (6,-4) -- (6, -5);
            
            \end{tikzpicture}
            \end{center}
            \end{onlyenv}

            \begin{onlyenv}<3>
            \begin{center}
            \begin{tikzpicture}
            [scale=0.35,auto=left,every node/.style={circle}]%,fill=blue!20}]
              \node [inode, color=black, label={[label distance=-5pt]90:\small\sffamily\bfseries A}](a) at (1, 7) {};
              \node [inode, color=blue, label={[label distance=-5pt]90:\small\sffamily\bfseries E}](b) at (3, 7) {};
              \node [inode, color=black, label={[label distance=-5pt]90:\small\sffamily\bfseries C}](c) at (5, 7) {};
              \node [inode, color=blue, label={[label distance=-5pt]90:\small\sffamily\bfseries D}](d) at (7, 7) {};
              \node [inode, color=black, label={[label distance=-5pt]90:\small\sffamily\bfseries B}](e) at (9, 7) {};
              \node [inode, color=blue, label={[label distance=-5pt]90:\small\sffamily\bfseries F}](f) at (11, 7) {};

              \draw [very thick] (a) -- (3,5) -- (4,6) -- (c);
              \draw [very thick] (4,6) -- (3.5, 6.5);
              \draw [very thick, draw=blue] (3.5, 6.5) -- (b);
              \draw [very thick, draw=blue] (d) -- (9,5) -- (10,6) -- (f);
              \draw [very thick, draw=blue] (10,6) -- (9.5, 6.5);
              \draw [very thick] (9.5, 6.5) -- (e);
              \draw [very thick] (3,5) -- (6,-4) -- (7.5, 0.5);
              \draw [very thick, draw=blue] (7.5, 0.5) -- (9, 5);
              \draw [very thick] (6,-4) -- (6, -5);
            
            \end{tikzpicture}
            \end{center}
            \end{onlyenv}
    \end{columns}
\end{frame}

% what is phylogenetics
% why phylogenetics is important
%     nothing in biology dob quote
%     Seen in the light of evolution, biology is, perhaps, intellectually the
%     most satisfying and inspiring science. Without that light it becomes a pile
%     of sundry facts some of them interesting or curious but making no
%     meaningful picture as a whole.
%     Dobzhansky, T. (1973). Nothing in biology makes sense except in the light of evolution. The American Biology Teacher 35:125--129.
%     ... nothing in evolution makes sense except in the light of phylogeny... SSB (http://systbio.org/teachevolution.html)
%     impossible to understand biodiversity without framework
%     biogeog
%         ecological biogeog -- independent contrast stuff
%         understanding diversification across space and time

% Classification before phylogeny -- been classifying stuff since hg days -> tends to form hiearchies
% Linneaus formalized this
% Evolution gave the framework behind why hiearchies were intuitive

%TERMINOLOGY
% Tree anatomy stuff
    % branches tips nodes etc
    % monophyly poly para etc.
    % branch rotation

% character terms
    % homology (shared due to no change) / homoplasy ("same change" has more than once; shared due to changes)
        % homologous
            % plesio / apo morphy
            % synapo /symplesio
        
% History -- three schools
    % Evol systematics
        % key point: very subjective and recognized paraphyletic groups
        % was very popular and legacy remains... many paraphyletic groups (fish, reptiles)
    
    % Numerical phenetics/taxonomy
        % sought objective, quantitative methods.
        % quantitiative clustering algorithms cluster taxa based on similarity of phenotypic traits.
        % resulted in paraphyletic/polyphyletic groups due to similarity among convergence
        % math / algorithms produced are still very useful for multivariate stats

    % Phylogenetic systematics / Cladistics
        % hennigian logic (mth lec 4)
        % logical inference: premises -> rules -> statement
        % premise: we can code characters
        % taxa that share a character state must be more closely related
        % Brief example
        % Later was parsimony method
        % no logical way to account for homoplasies -- need an error model!

% Methods
    % not going to dwell on Hennig... important then... not so much now.
    % Current methods are statistical (model-based).

% Molecular phylogenetics
    % Early indirect approaches to measuring changes in DNA
        % immunological assays
        % protein electrophoresis
        % DNA-DNA hybridization
        % restriction enzymes
    % PCR and sequencing game changer

% Molecular evolution
    % DNA - 4 character states
        % redundancy of code
        % transition/transversion bias
        % very conducive to modeling
        % different rates across genome -- conducive for resolving trees at different timescales
    % Phylogenomics
    % total evidence
    % morph character mapping

% Phylogeography
    % DNA data allowed for constructing trees of individual organisms (or gene copies) within and among species
    % gene trees
    % gene copies haplotypes
    % lineage sorting
    % coalescence
    % reciprocal monophyly
    % haplotype networks
    % gene trees within species trees (div times and incongruence)
    % book highlights these as problems, but we now have models of gene trees within species trees
        % "problems" can be a benefit for estimating historical demography and geography

% ecol niche modeling.
    % means of indirectly estimating demographic changes/ distributional shifts



\end{document}

